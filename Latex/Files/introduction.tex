\subsection{Purpose}
The purpose of the Student\&Company (S\&C) Platform is to create a system that allows Students to find Internships to enhance their education and improve their curriculum, while allowing Companies to find suitable candidates for their internship programs. All of this is done in a simple and efficient way by providing a series of tools to help both parties in the process.\\
S\&C will support the entire lifecycle of the Internship process for both Students and Companies: from the initial matchmaking that can be done automatically by the system through a proprietary Recommendation Process, or obtained by a Student with a Spontaneous Application to a specific internship offer, to the final selection process done through structured interviews created and submitted by Companies directly on the platform.\\% 
In the meantime, Student\&Company will also provide a series of Suggestions to improve CVs published by Students and internship offers published by Companies. The platform will also allow the Universities of students who are actually doing an internship to monitor the progress of such activities and handle any Complaints if necessary.
\subsubsection{Goals}


\subsection{Scope}

\subsubsection{World Phenomena}

\subsubsection{Shared Phenomena}


\subsection{Definitions, Acronyms, Abbreviations}  
This section offers explanations of terminology to elucidate the terms, acronyms and abbreviations used throughout the document, facilitating easy comprehension and reference for the readers.
\subsubsection{Definition}
\begin{itemize}
    \item \textbf{University}\label{def:university}: An institution that is registered on the S\&C platform.
    \item \textbf{Company}\label{def:compnay}: A company that is registered on the S\&C platform.
    \item \textbf{Student}\label{def:student}: A person who is currently enrolled in a University and is registered on the S\&C platform.
    \item \textbf{User}\label{def:user}: Any registered entity on the S\&C platform.
    \item \textbf{Internship}\label{def:internship}: The offer of a position provided by a company to one or more students. The position for a single student is temporary, but the offer remains active until it is removed from the platform.
    \item \textbf{Recommendation Process}\label{def:recommendationProcess}: The process of matching a Student with an Internship offered by a Company based on the Student's CV and the Internship's requirements made by the S\&C platform.
    \item \textbf{Recommendation/Match}\label{def:match}: The result of the Recommendation Process. It is the match between a Student and an Internship.
    \item \textbf{Spontaneous Application}\label{def:spontaneousApplication}: The process of a Student manually applying for an Internship that was not matched through the Recommendation Process.
    \item \textbf{Template Interview}\label{def:templateInterview}: A collection of open-ended questions, quizzes, and calls that a Company can create. Each time a Company wants to interview a Student, they can select a Template Interview, or create a new one, and assign it to the Student.
    \item \textbf{Interview}\label{def:Interview}: The process of evaluating a Student's application for an Internship, done through a Template Interview.
    \item \textbf{Feedback}\label{def:Feedback}: Information provided by Students and Companies to the S\&C platform to improve the Recommendation Process.
    \item \textbf{Suggestion}\label{def:suggestion}: Information provided by the S\&C platform to Students and Companies to improve their CVs and Internship descriptions.
    \item \textbf{Complaint}\label{def:complaint}: A report of a problem or issue that a Student or Company has with an ongoing Internship. It can be published on the platform and handled by the University.
\end{itemize}

\subsubsection{Acronyms}

\subsubsection{Abbreviations}


\subsection{Revision History}
\begin{table}[h]
    \centering
    \begin{tabular}{|l|c|l|}
        \hline
        Revised on & Version & Description\\ \hline
        ?-?-2024 & 1.0     & Initial Release of the document \\ \hline
    \end{tabular}
    \caption{Document Revision History}
    \label{tab:revision_history_table}
\end{table}

\subsection{Reference Documents}
\begin{itemize}
  \item Assignment RDD AY 2024-2025
  \item Software Engineering 2 A.Y. 2024/2025 Slides “CreatingRASD”
  \item IEEE Software Requirements Specification Template
  \item Alloy Documentations: \url{alloy.readthedocs.io}
\end{itemize}


\subsection{Document Structure}
\begin{enumerate}
    \item \textbf{Introduction}: This section provides an overview of the document and the system. Here the purpose of the platform is explained, along with the goals and phenomena of the system. Finally, essential definitions are provided.
    \item \textbf{Overall Description}: In this section, a high-level perspective of the system is provided, describing its overall purpose, functionality, and \hyperref[def:user]{User} interactions. It includes an outline of the system's intended features, user profiles, and assumptions about the domain.
    \item \textbf{Specific Requirements}: In this section, we focus on the technical and functional details of the system. Here, the external interfaces are specified as well as the functional and non-functional requirements of the system. Diagrams, such as use case and sequence diagrams, have been used to provide a visual representation of the system's functionality.
    \item \textbf{Alloy}: This section illustrates code and diagrams of the Alloy formal specification language that has been used to ensure the consistency and correctness of the system's formalized requirements.
    \item \textbf{Effort Spent}: This section provides an overview of the time spent by each group member on the project.
    \item \textbf{References}: This section provides a list of references used in the document.
\end{enumerate}