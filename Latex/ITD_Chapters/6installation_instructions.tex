\section{Installation Guide}

This section provides the necessary steps to install and run the S\&C platform.

\subsection{Prerequisites}

Before proceeding with the installation, ensure you have the following dependencies installed on your system:

\begin{itemize}
    \item \textbf{Docker}: required to run the backend services with the \verb|docker-compose.yml| file.
    \item \textbf{PowerShell} (Windows users only): optional, but recommended to automate the backend startup process.
    \item \textbf{Node.js and npm}: Required to install and run the frontend application.
\end{itemize}


\subsection{Backend Setup}
\subsubsection*{Docker}

\begin{enumerate}
    \item Ensure Docker is installed and running on your system.
    \item Open a terminal and navigate to the project root directory.

\end{enumerate}
\subsubsection*{Spring .jar Files \& Containerized Environment }

A PowerShell script is provided to create the \verb|.jar| files that will be needed to properly execute the backend services within the docker containers. 

\begin{enumerate}
    \item Open a PowerShell terminal.
    \item Navigate to the "Implementation" directory.
    \item Execute the script: 
    \begin{verbatim}
        .\build-and-execute.ps1
    \end{verbatim}

    \item When prompted, press \textbf{p} for production mode.
    \item Confirm container rebuild by pressing \textbf{yes} when asked.
\end{enumerate}

The script is useful as a shorthand to create the java build files, set up the environment variables and build and execute the containers.

Naturally, the same behavior could be obtained by manually compiling the Maven Java project inside the folders \verb|sc_server| and \verb|sc_auth| with the command \verb|mvn clean package| and then building the containers with \verb|docker compose up --build|



\subsection{Frontend Setup}

\begin{enumerate}
    \item Ensure that \textbf{Node.js} and \textbf{npm} are installed on your system.
    \item Open a terminal and navigate to the frontend directory.
    \item Install dependencies by running:

    \begin{verbatim}
    npm install
    \end{verbatim}

    \item Start the frontend with API support:

    \begin{verbatim}
    npm run dev:api
    \end{verbatim}

\end{enumerate}

\subsection{Final Steps}

Once both the backend and frontend are running, the application should be accessible via the specified local development URL. Make sure both services are properly started before testing the platform.

