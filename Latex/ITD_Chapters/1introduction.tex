\subsection{Purpose}
The purpose of the Student\&Company (S\&C) platform is to enable students to enroll into internships that will enhance their education and strengthen their CVs, while letting companies publish internship offers and select the best candidates through interviews. More over, S\&C allow students' universities to monitor each of their students' progress and intervene if needed.
The platform support and aid the users throughout the entire process by provide suggestion to the uploaded CVs and internship offers, automatically matches students and companies thanks to a proprietary algorithm, manage the distribution and collection of interviews and provides a space for filing and resolving complaints. The reader can find more information about the platform in the RASD document. 
In the remaining part of this chapter we will present a summary of the technical choices made for the creation of the platform and different bullet point lists and table including the Goals that we are trying to accomplish with this software and the Definition, Acronyms, Abbreviations used in this document.
\subsection{Scope}
This document, Implementation and Test Document (ITD), provides a comprehensive description of the implementation and testing phases of the S\&C platform. Specifically, it focuses on the functionalities developed, the adopted frameworks, and the structure of the source code. Additionally, it includes a detailed testing strategy, covering the procedures, tools, and methodologies used during the development process. This document also serves as a guide for installing and running the platform, offering installation instructions and addressing any prerequisites or potential issues. The effort spent by the team members is also summarized to provide insight into the workload distribution.

\subsection{Definitions, Acronyms, Abbreviations} 
This section provides definitions and explanations of the terms, acronyms, and abbreviations used throughout the document, making it easier for readers to understand and reference them.
\subsubsection{Definition}
The definition shared between this document and the RASD document are reported in the following list:
\begin{itemize}
    \item \textcolor{titleColor}{\textbf{University}\label{def:university}}: A university that is registered on the S\&C platform.
    \item \textcolor{titleColor}{\textbf{Company}\label{def:company}}: A company that is registered on the S\&C platform.
    \item \textcolor{titleColor}{\textbf{Student}\label{def:student}}: A person who is currently enrolled in a University and is registered on the S\&C platform.
    \item \textcolor{titleColor}{\textbf{User}\label{def:user}}: Any registered entity on the S\&C platform.
    \item \textcolor{titleColor}{\textbf{Internship Offer}\label{def:internshipOffer}}: The offer of an opportunity to enroll in an internship provided by a Company. The offer remains active on the platform indefinitely until the publishing Company removes it
    \item \textcolor{titleColor}{\textbf{Participant}}\label{def:participant}:{A Participant is an entity that interacts with the platform for the purpose of find or offering an Internship Position Offer, like Students and Companies
    }
    \item \textcolor{titleColor}{\textbf{Recommendation Process}}\label{def:recommendationProcess}: The process of matching a Student with an Internship offered by a Company based on the Student's CV and the Internship's requirements made by the S\&C platform.
    \item \textcolor{titleColor}{\textbf{Recommendation/Match}\label{def:match}}: The result of the Recommendation Process. It is the match between a Student and an Internship.
    \item \textcolor{titleColor}{\textbf{Spontaneous Application}\label{def:spontaneousApplication}}: The process of a Student spontaneously applying for an Internship that was not matched through the Recommendation Process.
    \item \textcolor{titleColor}{\textbf{Interview}\label{def:Interview}}: The process of evaluating a Student's application for an Internship done by a Company through the S\&C platform. 
    \item \textcolor{titleColor}{\textbf{Feedback}\label{def:Feedback}}: Information provided by Participant to the S\&C platform to improve the Recommendation Process.
    \item \textcolor{titleColor}{\textbf{Internship Position Offer}\label{def:internshipPositionOffer}}: The formal offer of an internship position presented to a student who has successfully passed the Interview, who can decide to accept or reject it.
    \item \textcolor{titleColor}{\textbf{Suggestion}\label{def:suggestion}}: Information provided by the S\&C platform to Participant to improve their CVs and Internship descriptions.
    \item \textcolor{titleColor}{\textbf{Confirmed Internship}\label{def:confirmdInternship}}: An Internship that has been accepted by the Student and the offering Company.
    \item \textcolor{titleColor}{\textbf{Ongoing Internship}\label{def:ongoing}}: A internship that is currently in progress. All Ongoing Internships are Confirmed Internships, but the vice versa is not always true.
    \item \textcolor{titleColor}{\textbf{Complaint}\label{def:complaint}}: A report of a problem or issue that a Student or Company has with an Ongoing Internship. It can be published on the platform and handled by the University.
    \item \textcolor{titleColor}{\textbf{Confirmed Match}\label{def:confirmedMatch}}: A match that has been accepted by both a Student and a Company.
    \item \textcolor{titleColor}{\textbf{Rejected Match}\label{def:rejectedMatch}}: A match that has been refused by either a Student or a Company.
    \item \textcolor{titleColor}{\textbf{Pending Match}\label{def:pendingMatch}}: A match that has been accepted only by a Student or a Company, waiting for a response from the other party.
    \item \textcolor{titleColor}{\textbf{Unaccepted Match}\label{def:unacceptedMatch}}: A match that has been refused by either a Student or a Company.
\end{itemize}

\subsubsection{Acronyms}
The following acronyms are used throughout the document:
\begin{table}[H]
    \centering
    \begin{tabular}{|c|c|}
        \hline
        \textbf{Acronym} & \textbf{Definition} \\ \hline
        ITD       & Implementation and Test Document  \\ \hline
        CV         & Curriculum Vitae \\ \hline
        UI         & User Interface \\ \hline
        UX         & User Experience \\ \hline
        DB         & Database \\ \hline
        API        & Application Programming Interface \\ \hline
        ORM        & Object-Relational Mapping \\ \hline
        DBMS       & Database Management System \\ \hline
        SPA        & Single Page Application \\ \hline
        DMZ        & Demilitarized Zone \\ \hline
        JPA        & Java Persistence API \\ \hline
        JS         & JavaScript \\ \hline
        JWT        & JSON Web Token \\ \hline
        HTTP       & HyperText Transfer Protocol \\ \hline
        HTTPS      & HyperText Transfer Protocol Secure \\ \hline
        SQL        & Structured Query Language \\ \hline
        CRUD       & Create, Read, Update, Delete \\ \hline
    \end{tabular}
    \caption{ITD Acronyms}
    \label{tab:ITDacronyms}
\end{table}

\subsubsection{Abbreviations}
The following abbreviations are used throughout the document:
\begin{table}[H]
    \centering
\begin{tabular}{|c|c|}
        \hline
        \textbf{Abbreviations} & \textbf{Definition} \\ \hline
        S\&C & Students\&Companies \\ \hline
    \end{tabular}
    \caption{RASD Abbreviations}
    \label{tab:abbreviations}
\end{table}
\subsection{Revision History}
\begin{table}[H]
    \centering
    \begin{tabular}{|c|c|c|}
        \hline
        \textbf{Revised on} & \textbf{Version} & \textbf{Description}\\ \hline
        2-2-2025 & 1.0     & Initial release of the document \\ \hline
    \end{tabular}
    \caption{Document Revision History}
    \label{tab:revision_history_table}
\end{table}